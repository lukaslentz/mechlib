\documentclass[border=10pt]{standalone}
\usepackage{amsmath}
\usepackage{siunitx}
\usepackage{xcolor}
\usepackage{tikz}
\usetikzlibrary{calc}


% Deine externe Datei einbinden
\input{..//..//mechlib.tex}

\begin{document}
	
	\begin{tikzpicture}
		% 0. Koordinaten
		\coordinate (A) at (0,0);
		\coordinate (B) at (0,-5);
		\coordinate (C) at (2,-5);
		
		% 1. Balken
		\pic (b1) {beam={A,B,0.4cm}};
		\pic (b2) {beam={b1-end-right,C,0.4cm}};
		
		% 2. Dimensionen
		\pic{idim={A,B,{$l_1$},-0.4cm}};
		\pic{idim={B,C,{$l_2$},-0.4cm}};
		
		% 3. Lager
		\pic[rotate=180] at (A) {fixed={1cm,0.2cm,45}};

		% 4. Lasten
		\pic{f={C,0,1cm,225,{$F_1$}}}; % Kraft
		\pic at (C) {adim={0,45,$\alpha$,0.5cm}}; % Winkel
		\draw[dashed] (C) -- ++(1,0); % Horizontalelinie
	\end{tikzpicture}
	
\end{document}