\documentclass[border=10pt]{standalone}
\usepackage{amsmath}
\usepackage{siunitx}
\usepackage{xcolor}
\usepackage{tikz}
\usetikzlibrary{calc}


% Deine externe Datei einbinden
\input{..//..//mechlib.tex}

\begin{document}
	
	\begin{tikzpicture}
		% 0. Koordinaten
		\coordinate (A) at (0,0);
		\coordinate (B) at (4,0);
		\coordinate (C) at (6,0);
		
		% 1. Balken
		\pic (b1) {beam={A,C,0.4cm}};
		
		% 2. Dimensionen
		\pic{idim={A,B,{$l_1$},-0.7cm}};
		\pic{idim={B,C,{$l_2$},-0.7cm}};
		
		% 3. Lager
		\pic[rotate=180] at (b1-start-right) {roller={1cm,0.2cm,45}};
		\pic[rotate=180] at (b1-end-right) {pined={1cm,0.2cm,45}};

		% 4. Lasten
		\pic{dl_uniform={B,A,0.2cm,1cm,$q_0$}}; % Last 1
		\pic{dl_linear={C,B,0.2cm,1.5cm,$q_1$}}; % Last 1
	\end{tikzpicture}
	
\end{document}