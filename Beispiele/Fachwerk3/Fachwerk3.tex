\documentclass[border=10pt]{standalone}
\usepackage{amsmath}
\usepackage{siunitx}
\usepackage{xcolor}
\usepackage{tikz}

\input{..//..//mechlib.tex}

\begin{document}
	
	\begin{tikzpicture}
		% Knoten anlegen
		\coordinate (K1) at (0,0);
		\coordinate (K2) at (0,2);
		\coordinate (K3) at (2,2);
		\coordinate (K4) at (4,2);
		\coordinate (K5) at (6,2);
		\coordinate (K6) at (4,4);
		\coordinate (K7) at (2,4);
		\makeJoints{K1,K2,K3,K4,K5,K6,K7}
		
		% Stäbe anlegen
		\pic{rod={K1,K2, 1, east,0}};
		\pic{rod={K1,K3, 2, north west, 1mm}};	
		\pic{rod={K2,K3, 3, south,0}};
		\pic{rod={K2,K7, 4, south east,1mm}};
		\pic{rod={K7,K3, 5, east,0}};
		\pic{rod={K3,K4, 6, south, 0}};
		\pic{rod={K3,K6, 7, south east, 1mm}};
		\pic{rod={K6,K7, 8, south,0}};
		\pic{rod={K6,K4, 9, east,0}};
		\pic{rod={K4,K5, 10, north,0}};
		\pic{rod={K5,K6, 11, south west, 1mm}};
		
		% Lager
		\pic[rotate=180] at (K1) {pined={1cm,1cm,45}};
		\pic[rotate=90] at (K2) {roller={1cm,1cm,45}};
		
		% 3. Last 
		\pic{f={K6,1,0.8cm,0,$F$}};
		\pic{f={K5,1,0.8cm,270,$F$}};
		
		% 4. Dimensionen
		\pic{idim={K2,K3,{$l$},-2.5cm}};
		\pic{idim={K3,K4,{$l$},-2.5cm}};	
		\pic{idim={K4,K5,{$l$},-2.5cm}};		
		\coordinate (KL) at (0,4);
		\pic{idim={K1,K2,{$l$},-0.8cm}};
		\pic{idim={K2,KL,{$l$},-0.8cm}};
	\end{tikzpicture}
	
\end{document}