\documentclass[border=10pt]{standalone}
\usepackage{amsmath}
\usepackage{siunitx}
\usepackage{xcolor}
\usepackage{tikz}

\input{..//..//mechlib.tex}

\begin{document}
	
	\begin{tikzpicture}
		\coordinate (K1) at (0,0);
		\coordinate (K2) at (2,0);
		\coordinate (K3) at (4,0);
		\coordinate (K4) at (2,1);
		\coordinate (K5) at (0,2);
		\makeJoints{K1,K2,K3,K4,K5}
			
		% Stäbe anlegen
		\pic{rod={K1,K2, 1, north,0}};
		\pic{rod={K2,K3, 2, north,0}};	
		\pic{rod={K3,K4, 3, south west, 1mm}};
		\pic{rod={K4,K5, 4, south west, 1mm}};
		\pic{rod={K2,K4, 5, east,0}};
		\pic{rod={K1,K4, 6, south east, 1mm}};
		\pic{rod={K1,K5, 7, east,0}};
		
		% Lager
		\pic[rotate=180] at (K1) {pined={1cm,1cm,45}};
		\pic[rotate=90] at (K5) {roller={1cm,1cm,45}};
		
		% 3. Last 
		\pic{f={K3,1,0.8cm,270,$F$}}; % Kraft
		\pic at (K3) {adim={180,155,$\alpha$,1cm}}; % Winkel
		
		% 4. Dimensionen
		\pic{idim={K1,K2,{$l_1$},-1.5cm}};
		\pic{idim={K2,K3,{$l_1$},-1.5cm}};		
		\coordinate (KL) at (0,1);
		\pic{idim={K1,KL,{$l_2$},-0.8cm}};
		\pic{idim={KL,K5,{$l_2$},-0.8cm}};
	\end{tikzpicture}
	
\end{document}