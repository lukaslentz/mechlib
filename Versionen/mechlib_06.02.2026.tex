% =========================================================
% TikZ / Mathe / Farben – Stylesammlung
% (Formatierung aufgeräumt, kommentiert; Inhalt unverändert)
% =========================================================

% --- Mathe / Einheiten / Farben ---
\usepackage{amsmath}
\usepackage{siunitx}
\usepackage{xcolor}

% --- TikZ Libraries ---
\usetikzlibrary{
	arrows.meta,
	calc,
	patterns,
	patterns.meta,
	intersections,
	angles,
	quotes,
	shapes.geometric,
	shadings
}

% --- Farbschema ---
\definecolor{ucbblue}{RGB}{6,13,141}
\definecolor{ucbgreen}{RGB}{67,176,42}
\definecolor{mmdblue}{rgb}{0.2,0.2,0.7}
\definecolor{tured}{rgb}{0.7,0.1,0.1}

% --- Standardfont in TikZ-Bildern ---
\tikzset{
	every picture/.style={font=\fontsize{10}{10}\selectfont}
}

% --- Standard-Schrift: serifenlos (Text im Dokument) ---
\renewcommand{\familydefault}{\sfdefault}


% =========================================================
% Mechanical Elements
% =========================================================

% ------------------ Farbstile ------------------
% Körper / Bauteile
\tikzset{
	body/.style={
		black,
		fill=gray!40
	},
}

% Lager / Auflager
\tikzset{
	support/.style={
		black,
		fill=gray!30
	},
}

% Lasten (distributed loads)
\tikzset{
	dl/.style={
		red
	},
	dl_fill/.style={
		opacity=0.2,
		red!50
	},
}

% ------------------ Ball ------------------
% Pic: ball
% Parameter: #1 Radius, #2 Farbe
\tikzset{
	pics/ball/.style args={#1,#2}{%
		code={
			\shade[ball color=#2, shading angle=0, draw=none] (0,0) circle [radius=#1];
		}
	},
	pics/ball/.default={0.5,red},
}

% ------------------ Beam ------------------
% Pic: beam
% Parameter: #1 Punkt A, #2 Punkt B, #3 Höhe (Offset in y)
\tikzset{
	pics/beam/.style args={#1,#2,#3}{%
		code={
			\path let \p1=(#1), \p2=(#2) in
			coordinate (-A) at (\x1, \y1)
			coordinate (-B) at (\x2, \y2 + #3);
			\draw[body] (-A) rectangle (-B);
		}
	},
	pics/beam/.default={(0,0),(1,0),0.3},
}


% =========================================================
% Supports
% =========================================================

% ------------------ Roller (Loslager) ------------------
% Pic: roller
% Parameter: #1 Breite, #2 Höhe, #3 Schraffurwinkel
\tikzset{
	pics/roller/.style args={#1,#2,#3}{
		code={
			\def\mlw{0.2cm} \def\mlh{0.2cm} \def\mlr{0.06cm}
			% Lagerdreieck
			\draw[support] (0,0) -- (-0.5*\mlw, \mlh-1pt) --+ (\mlw,0) -- cycle;
			\draw[support] (0,0) circle (\mlr);
			
			% Schraffierte Bodenplatte
			\fill[
			pattern={Lines[angle=#3, distance={1pt}]},
			pattern color=black
			] (-0.5*\mlw,\mlh) rectangle (0.5*\mlw,1.4*\mlh);
			
			\draw (-0.5*\mlw, \mlh) -- (0.5*\mlw, \mlh);
		}
	},
	pics/roller/.default = {0.3cm,0.2cm,45},
}

% ------------------ Pined (Gelenklager) ------------------
% Pic: pined
% Parameter: #1, #2 (aktuell nicht genutzt), #3 Schraffurwinkel
\tikzset{
	pics/pined/.style args={#1,#2,#3}{
		code={
			\def\mlw{0.2cm}
			\def\mlh{0.2cm}
			\def\mlr{0.06cm}
			\draw[support] (0,0) -- (-0.5*\mlw, \mlh) --+ (\mlw,0) -- cycle;
			\draw[support] (0,0) circle (\mlr);
			
			\fill[
			pattern={Lines[angle=#3, distance={1pt}]},
			pattern color=black
			] (-0.5*\mlw,\mlh) rectangle (0.5*\mlw,1.4*\mlh);
		}
	},
	pics/pined/.default = {1cm,1cm,45},
}

% ------------------ Fixed Support (Einspannung) ------------------
% Pic: fixed
% Parameter: #1 Breite, #2 Höhe, #3 Schraffurwinkel
\tikzset{
	pics/fixed/.style args={#1,#2,#3}{
		code={
			\def\mlw{#1}
			\def\mlh{#2}
			\draw[] (-0.5*\mlw,0) -- (0.5*\mlw,0);
			\fill[
			pattern={Lines[angle=#3, distance={3pt/sqrt(2)}]},
			pattern color=black
			] (-0.5*\mlw,0) rectangle (0.5*\mlw,-\mlh);
		}
	},
	pics/fixed/.default = {1cm,0.1cm,45},
}


% =========================================================
% Loads
% =========================================================

% ------------------ Point Force ------------------
% Pic: f
% Parameter: #1 Bezugspunkt, #2 (aktuell nicht genutzt)
% (Hinweis: Der Draw-Befehl ist hier wie im Original unvollständig.)
\tikzset{
	pics/f/.style args={#1,#2}{
		code={
			\path let \p1=(#1) in
			coordinate (F) at (\x1, \y1);
			\draw[-latex,red]
		}
	},
	pics/f/.default = {0.7cm,0.3cm},
}

% ------------------ Uniformly Distributed Load ------------------
% Pic: dl_uniform
% Parameter: #1 Punkt A, #2 Punkt B, #3 Offset in y, #4 Höhe der Lastdarstellung
\tikzset{
	pics/dl_uniform/.style args={#1,#2,#3,#4}{
		code={
			\path let \p1=(#1), \p2=(#2) in
			coordinate (-A) at (\x1, \y1 + #3)
			coordinate (-B) at (\x2, \y2 + #3);
			
			\fill[dl_fill] (-A) rectangle ($(-B) + (0,#4)$);
			
			\draw[dl]
			($(-A) + (0,.5*#4)$) -- ($(-A) + (0,#4)$) -- ($(-B) + (0,#4)$) -- ($(-B) + (0,.5*#4)$);
			
			\foreach \dx in {0,0.2,...,1}{
				\draw[-latex,dl] ($(-A)!\dx!(-B) + (0,#4)$) -- +(0,-#4);
			}
			
			\node[anchor=south,dl] at ($(-B) + (0,#4)$) {$q_0$};
		}
	},
	pics/dl_uniform/.default = {(0,0),(1,0),0.3cm,0.8cm},
}

% ------------------ Linearly Distributed Load ------------------
% Pic: dl_linear
% Parameter: #1 Punkt A, #2 Punkt B, #3 Offset in y, #4 Maximalhöhe
\tikzset{
	pics/dl_linear/.style args={#1,#2,#3,#4}{
		code={
			\path let \p1=(#1), \p2=(#2) in
			coordinate (-A) at (\x1, \y1 + #3)
			coordinate (-B) at (\x2, \y2 + #3);
			
			\fill[opacity=0.2] (-A) -- ($(-B) + (0,#4)$) -- (-B) -- cycle;
			
			\draw (-A) -- ($(-B) + (0,#4)$) -- ($(-B) + (0,.5*#4)$);
			
			\foreach \dx in {0.2,0.4,...,1}{
				\draw[-latex] ($(-A)!\dx!(-B) + (0,\dx*#4)$) -- +(0,-\dx*#4);
			}
			
			\node[anchor=south] at ($(-B) + (0,#4)$) {$q_0$};
		}
	},
	pics/dl_linear/.default = {(0,0),(1,0),0.3cm,0.8cm},
}


% =========================================================
% Dimensions
% =========================================================

% ------------------ Arc Dimension ------------------
% Pic: adim
% Parameter: #1 Startwinkel, #2 Endwinkel, #3 Text, #4 Radius
\tikzset{
	pics/adim/.style args={#1,#2,#3,#4}{%
		code={
			\def\mlr{#4}
			\draw[] (#1:\mlr) arc (#1:#2:\mlr);
			\node[] at (.5*#1 + .5*#2 :.7*\mlr){#3};
		}
	},
	pics/adim/.default={0,45,{$\alpha$},.7cm},
}

% ------------------ Horizontal Dimension ------------------
% Pic: hdim
% Parameter: #1 Punkt A, #2 Punkt B, #3 Text, #4 Offset in y
\tikzset{
	pics/hdim/.style args={#1,#2,#3,#4}{%
		code={
			\path let \p1=(#1), \p2=(#2) in
			coordinate (-A) at (\x1, \y1 + #4)
			coordinate (-B) at (\x2, \y2 + #4);
			
			\draw[latex-latex, line width=0.5pt] (-A) -- (-B)
			node[pos=0.5, below]{#3};
		}
	},
	pics/hdim/.default={(0,0),(1,0),{$l$},0.3cm},
}


% =========================================================
% Coordinate Systems
% =========================================================

% ------------------ x/y Koordinatensystem ------------------
% Pic: xy_cos
% Parameter: #1 x-Länge, #2 y-Länge
\tikzset{
	pics/xy_cos/.style args={#1,#2}{
		code={
			\draw[latex-latex]
			(#1,0)--(0,0)node[anchor=north,pos=0]{$x$}
			--(0,#2)node[anchor=east,pos=1]{$y$};
		}
	},
	pics/xy_cos/.default = {0.8cm,0.8cm},
}

% ------------------ x/z Koordinatensystem ------------------
% (Hinweis: Original verwendet (0,-\r); \r ist hier nicht definiert – unverändert gelassen.)
\tikzset{
	cosys/.pic={
		\def\mlr{1cm}
		\draw[latex-latex]
		(0,-\r) -- (0,0) node[pos=0,anchor=north] {$z$}
		-- (\mlr,0) node[pos=1,anchor=west]  {$x$};
		
		\draw[-latex, shorten >=-2pt]
		(-80:0.8*\mlr) arc (-80:-20:0.8*\mlr)
		node[pos=0.5,anchor=north west]{+};
	}
}


% =========================================================
% Reaction Forces
% =========================================================

% ------------------ Reaktionsgrößen an Einspannung ------------------
% Pic: sf_fixed
% Parameter: #1 Pfeillänge (Skalierung), #2 Höhe (Referenz)
\tikzset{
	pics/sf_fixed/.style args={#1,#2}{
		code={
			\def\mlf{#1}
			\def\mlh{#2}
			
			\draw[line width=0.5pt,-latex,red]
			($(0,0.5*\mlh)+(-0.8*\mlf,0)$) -- +(0.8*\mlf,0)
			node[pos=0,anchor=east,xshift=.1cm] {$A_x$};
			
			\draw[line width=0.5pt,-latex,red]
			($(0,0.5*\mlh)+(-1pt,0.4*\mlf)$) -- +(0,-\mlf)
			node[pos=0,anchor=south,yshift=-.1cm,xshift=-1mm] {$A_z$};
			
			\draw[line width=0.5pt,-latex,red,shorten >=-2pt]
			($(0,0.5*\mlh)+(120:0.4*\mlf)$) arc (120:230:0.4*\mlf)
			node[pos=1,anchor=east,yshift=-.1cm,xshift=1mm] {$M_A$};
		}
	},
	pics/sf_fixed/.default = {0.7cm,0.3cm},
}


% =========================================================
% Internal Forces
% =========================================================

% ------------------ Schnittgrößen positiv ------------------
% Pic: sl_pos
% Parameter: #1 Pfeillänge, #2 Höhe (Referenz)
\tikzset{
	pics/sl_pos/.style args={#1,#2}{
		code={
			\def\mlf{#1}
			\def\mlh{#2}
			
			\draw[line width=0.5pt,-latex,red] (0,0.5*\mlh) -- +(0.8*\mlf,0)
			node[pos=1,anchor=west,xshift=-.1cm] {$N$};
			
			\draw[line width=0.5pt,-latex,red]
			($(0,0.5*\mlh) + (1pt,0.4*\mlf)$) -- +(0,-\mlf)
			node[pos=0,anchor=south,yshift=-.1cm,xshift=.1cm] {$Q$};
			
			\draw[line width=0.5pt,-latex,red,shorten >=-2pt]
			($(0,0.5*\mlh) + (-60:0.4*\mlf)$) arc(-60:50:0.4*\mlf)
			node[pos=1,anchor=west,yshift=.1cm,xshift=.05cm] {$M$};
		}
	},
	pics/sl_pos/.default = {0.7cm,0.3cm},
}

% ------------------ Schnittgrößen negativ ------------------
% Pic: sl_neg
% Parameter: #1 Pfeillänge, #2 Höhe (Referenz)
\tikzset{
	pics/sl_neg/.style args={#1,#2}{
		code={
			\def\mlf{#1}
			\def\mlh{#2}
			
			\draw[line width=0.5pt,-latex,red] (0,0.5*\mlh) -- +(-0.8*\mlf,0)
			node[pos=1,anchor=east,xshift=.1cm] {$N$};
			
			\draw[line width=0.5pt,-latex,red]
			($(0,0.5*\mlh)+(-1pt,-0.4*\mlf)$) -- +(0,\mlf)
			node[pos=1,anchor=south,yshift=-.1cm,xshift=-.1cm] {$Q$};
			
			\draw[line width=0.5pt,-latex,red,shorten >=-2pt]
			($(0,0.5*\mlh)+(240:0.4*\mlf)$) arc (240:130:0.4*\mlf)
			node[pos=1,anchor=east,yshift=.1cm,xshift=-.05cm] {$M$};
		}
	},
	pics/sl_neg/.default = {0.7cm,0.3cm},
}
